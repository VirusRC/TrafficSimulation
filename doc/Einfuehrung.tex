% !TeX root = Trafficsimulation_Documentation.tex

\chapter{Einführung}
\label{Einführung}

Im Abschnitt \ref{Einführung} wird dem Leser ein Überblick der Aufgaben des Verkehrssimulationsprojekts gegeben.

\thispagestyle{standard}
\pagestyle{standard}

\section{Aufgabenstellung}
\label{Aufgabenstellung}

Im Rahmen dieser Übung soll eine Verkehrssimulation realisiert werden. Dabei sollen sich diverse Verkehrsteilnehmer, zum Beispiel Autos und Busse, entsprechend der üblichen Straßenverkehrsregeln in einem gegebenen Straßennetz bewegen. Der Anwender der Simulation soll die Möglichkeit haben sowohl Simulationsparameter als auch Straßennetze modifizieren zu können. Für die Einstellung der Simulationsparameter soll eine Editor Oberfläche erstellt werden. Über diese Oberfläche hat der Benutzer die Möglichkeit vor und während der Simulation, Parameter wie die maximale Geschwindigkeit der Fahrzeuge, Beschleunigung der Fahrzeuge, Einfahrtsrate der Fahrzeuge oder Ampelschaltzeiten anzupassen. Die Simulation soll in einer zwei- oder dreidimensionalen graphischen Oberfläche dargestellt werden. Der Benutzer soll aus diversen Kartentypen, welche persistent gespeichert sein sollen, zu Beginn der Simulation auswählen können. 

In den weiteren Lehreinheiten wurden weitere Aufgaben definiert. So sollte eine gruppenübergreifende Kommunikation möglich sein, Autos sollen von einer Simulation in die nächste fahren können.

Eine weitere Aufgabenstellung war die Möglichkeit ein  Hinderniss in die Fahrbahn zu platzieren. Dies sollte frei möglich sein (also zur Laufzeit). Ein Auto soll dieses Hindernis umfahren können und gleichzeitig eine Kollision mit einem anderen Auto verhindern.

