% !TeX root = Trafficsimulation_Documentation.tex

\documentclass
[
a4paper,
german,
openright,                    % Kap.beginn immer rechts! (fkt. nur bei report, nicht bei article)
11pt                          % ersatzweise 12pt, wenn mehr Seiten entstehen sollen
]
{report}

%%%%%%%%%%%%%%%%%%%%%%%%%%%%%%%%%%%%%%%%%%%%%%%%%%%%%%%%%%%%%%%%%%%%%%%%%%%%%%%%%%%%%%%%%%%%%%%%%%%%%%%%%%%%
%Dokumenteneigenschaften

\newcommand{\Author}{Fabian Sch\"orghofer, Andreas Reschenhofer, Lukas Altenhuber, Paul Riedl, Mike Thomas}
\newcommand{\FS}{Fabian Sch\"orghofer}
\newcommand{\AR}{Andreas Reschenhofer}
\newcommand{\LA}{Lukas Altenhuber}
\newcommand{\PR}{Paul Riedl}
\newcommand{\MT}{Mike Thomas}
\newcommand{\Title}{Verkehrssimulation Dokumentation}
\newcommand{\Keywords}{Software-Architektur, Software, Dokumentation}
\newcommand{\Advisor}{DI (FH) DI Roland Graf, MSc}
\newcommand{\Birthdate}{13.11.1992}
\newcommand{\EnrolNum}{1610581022}
\newcommand{\VenueMonthYear}{Puch am, \today}
\newcommand{\VenueDate}{Salzburg, am \today}

%%%%%%%%%%%%%%%%%%%%%%%%%%%%%%%%%%%%%%%%%%%%%%%%%%%%%%%%%%%%%%%%%%%%%%%%%%%%%%%%%%%%%%%%%%%%%%%%%%%%%%%%%%%%
% Layout zusammengestellt von Richard Wanger im WS 2015/2016. Verschiedene Vorlagen wurden bei der Erstellung
% kombiniert und nach dem Masterleitfaden (Stand: Oktober 2015) adaptiert. Verwendung ohne Gewähr!

\usepackage{style}		%alle packages und Formatvorlagen befinden sich in dieser Datei
\usepackage{amsmath}
\usepackage{xcolor}
\usepackage{vhistory, hyperref}
\definecolor{light-gray}{gray}{0.95}
\lstdefinestyle{sharpc}{language=[Sharp]C, frame=single, rulecolor=\color{black!80!blue}, backgroundcolor=\color{light-gray}, numberstyle=\small\color{gray}, commentstyle=\itshape\color{green!40!black}, emph={var, get, set, value}, emphstyle={\color{blue}}, escapeinside={\%*}{*)}, keywordstyle=[2]\color{ 	cyan}, keywords=[2]{XmlArray, XmlArrayItem, List, PointLatLng, CategoryAttribute, DisplayName, Description, XmlIgnore, XmlElement, MouseEventArgs, Cursors, MouseButtons, GMapOverlay, GMapRoute, BackgroundWorker, File, Path, XmlDocument, DoWorkEventArgs, GISConfiguration, TextWriter, StreamWriter, TextReader, StreamReader, InvalidOperationException, Debug, GISControl, GISControlLayer, GISControlLine, XmlSerializer}}
\definecolor{dkgreen}{rgb}{0,0.6,0}
\definecolor{gray}{rgb}{0.5,0.5,0.5}
\definecolor{mauve}{rgb}{0.58,0,0.82}

\lstset{frame=tb,
  language=C,
  aboveskip=3mm,
  belowskip=3mm,
  showstringspaces=false,
  columns=flexible,
  basicstyle={\small\ttfamily},
  numbers=left,
  numberstyle=\tiny\color{gray},
  keywordstyle=\color{blue},
  commentstyle=\color{dkgreen},
  stringstyle=\color{mauve},
  breaklines=true,
  breakatwhitespace=true,
  tabsize=3
}

%%%%%%%%%%%%%%%%%%%%%%%%%%%%%%%%%%%%%%%%%%%%%%%%%%%%%%%%%%%%%%%%%%%%%%%%%%%%%%%%%%%%%%%%%%%%%%%%%%%%%%%%%%%%
% ORGANISATORISCHES

\begin{document}
% !TeX root = Trafficsimulation_Documentation.tex

\begin{titlepage}

\hspace{7cm}

\begin{center}
	{\Large\uppercase\expandafter{\bf Software-Architekturen}}\\[0.5ex]
	\vspace{1cm}
	\Large{\bf\large \Title}\\
	\vspace{1.5cm}
	\normalsize durchgeführt am\\
	Studiengang Informationstechnik \& System--Management\\
	an der\\
	Fachhochschule Salzburg GmbH\\
\end{center}

\vspace{2cm}

\begin{center}
	\normalsize vorgelegt von
	\\
	{
		\Large{\bf\large \Author}\\
	}
	\vspace{2cm}
	\includegraphics[width=7cm]{BilderAllgemein/Logo.jpg}\medskip
\end{center}
	
\vspace{2cm}

\begin{tabbing}
	\hspace*{3cm}\=\hspace*{5.5 cm}\= \kill
	\> Leiter des Studiengangs: \> FH-Prof.~DI Dr. Gerhard Jöchtl \\*[0.2cm]
	\> Betreuer: \> \Advisor \\
	\> Betreuer: \> DI Eduard Hirsch
\end{tabbing}

\vfill	

\begin{center}
\VenueMonthYear\\
\end{center}
\end{titlepage}
%\include{02EidesstattlicheErklaerung}
%\include{03InfosAbstract}

%%%%%%%%%%%%%%%%%%%%%%%%%%%%%%%%%%%%%%%%%%%%%%%%%%%%%%%%%%%%%%%%%%%%%%%%%%%%%%%%%%%%%%%%%%%%%%%%%%%%%%%%%%%%
%VERZEICHNISSE

\tableofcontents
%\protect \addcontentsline{toc}{chapter}{Inhaltsverzeichnis}

%\renewcommand{\nomname}{Abkürzungsverzeichnis}
%\include{05Abkuerzungsverzeichnis}

%\listoffigures
%\protect \addcontentsline{toc}{chapter}{Abbildungsverzeichnis}

%\listoftables
%\protect \addcontentsline{toc}{chapter}{Tabellenverzeichnis}

%\lstlistoflistings
%\protect \addcontentsline{toc}{chapter}{Listingverzeichnis}

%%%%%%%%%%%%%%%%%%%%%%%%%%%%%%%%%%%%%%%%%%%%%%%%%%%%%%%%%%%%%%%%%%%%%%%%%%%%%%%%%%%%%%%%%%%%%%%%%%%%%%%%%%%%
%INHALT

% !TeX root = Trafficsimulation_Documentation.tex

\chapter{Einführung}
\label{Einführung}

Im Abschnitt \ref{Einführung} wird dem Leser ein Überblick der Aufgaben des Verkehrssimulationsprojekts gegeben.

\thispagestyle{standard}
\pagestyle{standard}

\section{Aufgabenstellung}
\label{Aufgabenstellung}

Im Rahmen dieser Übung soll eine Verkehrssimulation realisiert werden. Dabei sollen sich diverse Verkehrsteilnehmer, zum Beispiel Autos und Busse, entsprechend der üblichen Straßenverkehrsregeln in einem gegebenen Straßennetz bewegen. Der Anwender der Simulation soll die Möglichkeit haben sowohl Simulationsparameter als auch Straßennetze modifizieren zu können. Für die Einstellung der Simulationsparameter soll eine Editor Oberfläche erstellt werden. Über diese Oberfläche hat der Benutzer die Möglichkeit vor und während der Simulation, Parameter wie die maximale Geschwindigkeit der Fahrzeuge, Beschleunigung der Fahrzeuge, Einfahrtsrate der Fahrzeuge oder Ampelschaltzeiten anzupassen. Die Simulation soll in einer zwei- oder dreidimensionalen graphischen Oberfläche dargestellt werden. Der Benutzer soll aus diversen Kartentypen, welche persistent gespeichert sein sollen, zu Beginn der Simulation auswählen können. 

In den weiteren Lehreinheiten wurden weitere Aufgaben definiert. So sollte eine gruppenübergreifende Kommunikation möglich sein, Autos sollen von einer Simulation in die nächste fahren können.

Eine weitere Aufgabenstellung war die Möglichkeit ein  Hinderniss in die Fahrbahn zu platzieren. Dies sollte frei möglich sein (also zur Laufzeit). Ein Auto soll dieses Hindernis umfahren können und gleichzeitig eine Kollision mit einem anderen Auto verhindern.



% !TeX root = Trafficsimulation_Documentation.tex

\chapter{Software-Architektur}
\label{Software-Architektur}

In diesem Kapitel wird die Architektur in Form des Arc42-Templates sowohl grafisch als auch textuell dargestellt. 

\thispagestyle{standard}
\pagestyle{standard}

\section{Einführung und Ziele}
\label{Einführung und Ziele}



\section{Randbedingungen}
\label{Randbedingungen}


\begin{flushleft}
\textbf{Vorgaben}
\end{flushleft}
\vspace{-0.3 cm}

Die Verwendung der Programmiersprache C\# und die Auslagerung der Logik für geregelte Kreuzungen sind als Vorgaben für die Realisierung der Verkehrssimulation gegeben.

\section{Kontextabgrenzung}
\label{Kontextabgrenzung}

\section{Lösungsstrategie}
\label{Lösungsstrategie}

\section{Bausteinsicht}
\label{Bausteinsicht}

\section{Laufzeitschicht}
\label{Laufzeitschicht}

\section{Verteilungssicht}
\label{Verteilungssicht}

In diesem Abschnitt folgt die Beschreibung der Komponenten.

\begin{figure}[H]
\begin{center}
	\includegraphics[scale=0.673]{BilderAllgemein/Komponentdiagram.JPG}
\end{center}
	%\includegraphics[width=\textwidth]
	%\end{center}
	% Title
	\caption{Komponenten Diagramm Verkehrssimulation}
	% Unique name: identifier for referencing
	%\label{Abbildung 2.1}
\end{figure}

\begin{flushleft}
\textbf{Verkehrssimulation und Ampelsteuerung}
\end{flushleft}
\vspace{-0.3 cm}

Sowohl Verkehrssimulation als auch Ampelsteuerung sind eigene Executables. Die Verkehrssimulation wird aus dem Unity-Projekt erzeugt und beinhaltet alle oben angegebenen Komponenten. Die Ampelsteuerung ist eine Konsolen Applikation, welche über die IPC-Komponenten mit der Verkehrssimulation kommuniziert.

\begin{flushleft}
\textbf{Verkehrsteilnehmer}
\end{flushleft}
\vspace{-0.3 cm}

Zu diesen Komponenten gehören alle sich aktiv in der Simulation bewegenden Objekte, wie Autos, Busse und LKWs. Die einzelnen Komponenten bewegen sich autonom voneinander im Straßennetz fort. Die Komponenten scannen ihre Umgebung auf Fahrbahnen, Verkehrsregeln und andere Verkehrsteilnehmer. Auf Basis dieses Scans entscheiden sie ihre nächste Aktion, wie zum Beispiel Beschleunigen, Bremsen oder Abbiegen. Verkehrsteilnehmer können sich ausschließlich auf Fahrbahnen bewegen und beachten alle Verkehrsregeln innerhalb ihres Aktionsradius. Weiters benötigen sie Simulationsparameter, die über ihre maximale Geschwindigkeit, Beschleunigung und Bremsverhalten entscheiden. Verkehrsteilnehmer treffen ihre Entscheidungen bei jedem Renderdurchlauf der Game Engine. Die Verknüpfung der Verkehrsteilnehmer mit der GUI erfolgt abstrahiert durch Unity.

\begin{flushleft}
\textbf{Dynamische Umgebungsobjekte}
\end{flushleft}
\vspace{-0.3 cm}

Dynamische Umgebungsobjekte sind Objekte, welche während der Simulationslaufzeit ihre Eigenschaften ändern, jedoch nicht ihre Position. Konkret sind dynamische Umgebungsobjekte Ampeln. Ampeln bieten den umliegenden Verkehrsteilnehmern ihren derzeitigen Status als Verkehrsregeln an, welche diese beachten müssen. Dynamische Elemente benötigen Simulationsparameter, welche ihre Schaltzeiten vorgeben. Konkret geben die Simulationsparameter die Phasenzeiten der Ampeln an. Weiters benötigen dynamische Umgebungsobjekte Steuerungskommandos um zwischen diversen Modi zu wechseln. Über Steuerungskommandos können Ampeln von automatischen Betrieb in einen inaktiven dauerhaft Gelb blinkenden Status gebracht werden. In jedem Renderzyklus der Game Engine wird auf neue Steuerungskommandos geprüft und falls nötig der vorgegebene Schaltvorgang eingeleitet. Die Verknüpfung der dynamischen Umgebungsobjekte mit der GUI erfolgt abstrahiert durch Unity.

\begin{flushleft}
\textbf{Simulationssteuerung}
\end{flushleft}
\vspace{-0.3 cm}

Die Simulationssteuerung stellt eine übergeordnete Kontrollinstanz der Simulation dar. Sie legt globale Einstellungen für Simulationskomponenten zentral fest. Die Simulationssteuerung bietet Simulationsparameter für andere Simulationskomponenten an, welche diese abrufen können. Konkrete Simulationsparameter sind Schaltzeiten für Ampeln, Höchstgeschwindigkeiten für Verkehrsteilnehmer oder Spawnraten für neue Verkehrsteilnehmer. Die Simulationssteuerung benötigt Inputparameter, welche von außen die Simulationsparameter bestimmen. Die Aktualisierung der Simulationsparameter auf Basis der Inputparameter erfolgt bei jedem Renderzyklus der Game Engine.

\begin{flushleft}
\textbf{Konfigurationseditor}
\end{flushleft}
\vspace{-0.3 cm}

Der Konfigurationseditor stellt eine zweidimensionale graphische Benutzeroberfläche dar, welche es dem Benutzer ermöglicht angebotene Inputparameter für die Simulation zu verändern. Vom Benutzer geänderte Inputparameter werden bei jedem Renderzyklus der Game Engine verarbeitet und entsprechend weiter geleitet.

\begin{flushleft}
\textbf{IPC-Peer Verkehrssimulation}
\end{flushleft}
\vspace{-0.3 cm}

Der IPC-Peer Verkehrssimulation repräsentiert eine Instanz der Inter Process Communication zwischen den Executables Verkehrssimulation und Ampelsteuerung dar. Dieser IPC-Peer gibt Initialisierungskommandos an die Ampelsteuerung weiter. Über diese Kommandos werden die benötigte Anzahl an Ampeln mit den korrekten Initialisierungsparametern angelegt. Weiters werden Ampelschaltkommandos entgegen genommen und weiter geleitet um den Modus einer Ampel zu wechseln. Ein- und ausgehende Kommandos während in jedem Renderzyklus der Game Engine bearbeitet.

\begin{flushleft}
\textbf{Statische Umgebungsobjekte}
\end{flushleft}
\vspace{-0.3 cm}

Statische Umgebungsobjekte repräsentieren Simulationskomponente, welche weder ihre Eigenschaften noch ihre Position während der Simulationslaufzeit verändern. Konkret sind Straßen und Verkehrsschilder statische Umgebungsobjekte. Sie bieten Fahrbahnen für die Verkehrsteilnehmer an, auf welchen diese sich bewegen können. Zusätzlich werden auch Verkehrsregeln vorgegeben, welche von den Verkehrsteilnehmer beachtet werden müssen. Die statischen Umgebungsobjekte werden von der Game Engine gerendert, jedoch sollte diese Komponente keine Logik besitzen. Die Verknüpfung der statischen Umgebungsobjekte mit der GUI erfolgt abstrahiert durch Unity.

\begin{flushleft}
\textbf{IPC-Peer Ampelsteuerung}
\end{flushleft}
\vspace{-0.3 cm}

Die Komponente IPC-Peer Ampelsteuerung bildet die Gegenstelle der zuvor beschriebenen IPC-Peer Verkehrssimulation. Sie gibt Ampelschaltkommandos zur Gegenstelle weiter und nimmt Initialisierungskommandos entgegen. Diese Kommandos werden intern, innerhalb der Ampelsteuerung Executable an die Ampelsteuerungskomponente weiter gegeben.

\begin{flushleft}
\textbf{Ampelsteuerungskomponente}
\end{flushleft}
\vspace{-0.3 cm}

Die Ampelsteuerungskomponente übernimmt die Verwaltung der einzelnen Ampelinstanzen. Anhand eingehender Initialisierungskommandos werden Ampelinstanzen mit den benötigten Initialisierungsparametern erstellt. Über das angebotene CLI kann der Benutzer Schaltbefehle absetzen, welche über Ampelschaltungskommandos weiter geleitet werden.

\begin{flushleft}
\textbf{Message Queue}
\end{flushleft}
\vspace{-0.3 cm}

Die Message Queue übernimmt die Kommunikation mit den anderen Gruppen. Die Kommunikation mit den anderen Gruppen wird über einen externen Server abgewickelt, auf denen sich die Gruppen anmelden und ihre Queues abonnieren können. Ein Format, basierend auf JSON wurde zwischen den Gruppenteilnehmern definiert.

Als Protokoll wird dabei AMPQ verwendet, der dazugehörende Server, RabbitMQ implementiert diese Protokoll und ermöglicht die Übertragung von Nachrichten.

\section{Querschnittliche Konzepte}
\label{Querschnittliche Konzepte}

\section{Entwurfsentscheidungen}
\label{Entwurfsentscheidungen}

\section{Qualitätsszenarien}
\label{Qualitätsszenarien}

\section{Risiken und technische Schulden}
\label{Risiken und technische Schulden}
% !TeX root = Trafficsimulation_Documentation.tex

\chapter{Design}
\label{Design}

In diesem Kapitel werden die Implementierungen der einzelnen Komponenten sowie deren Schnittstellen zu anderen Komponenten dokumentiert.

\thispagestyle{standard}
\pagestyle{standard}

\section{Strassensystem}
\label{Strassensystem}

\subsection{Ampeln}

\subsection{Strassen}

\subsection{Kreuzungen}

\section{Fahrzeuglogik}
\label{Fahrzeuglogik}

\section{Ampelsteuerung}

\section{Hindernisse}
\label{Hindernisse}

\section{Aus- und Einfahren von Fahrzeugen anderer Gruppen}
% !TeX root = Trafficsimulation_Documentation.tex

\chapter{Implementierung}
\label{Implementierung}

In diesem Kapitel werden die Implementierungen der einzelnen Komponenten sowie deren Schnittstellen zu anderen Komponenten dokumentiert.

\thispagestyle{standard}
\pagestyle{standard}

\section{Erstellung der Welt}
\label{Erstellung_der_Welt}

\section{Fahrzeugsteuerung}
\label{Fahrzeugsteuerung}
% Inklusive Abbiegen, Weg folgen, Spawn

\section{Erkennen und Reagieren auf Ampeln}
\label{Erkennen_und_Reagieren_auf_Ampeln}

\section{Fahrzeug Kollisionserkennung}
\label{Fahrzeug_Kollisionserkennung}

Jedes Fahrzeug implementiert das "FollowWay.cs" Skript welches die komplette Logik über das Verkehrsverhalten besitzt. Innerhalb dieses Skripts wird in der Update-Methode, welche in jedem Frame pro Sekunde aufgerufen wird, überprüft, ob ein Raycast mit einem anderen Fahrzeug oder Hindernis kollidiert ist. Da dieser Raycast mehrere Objekte durchdringen kann, muss in einer Liste überprüft werden ob mit einem dieser Objekte kollidiert werden soll. Ist dies der Fall, so wird die Geschwindigkeit des Fahrzeugs verringert und der Raycast wird auf die Länge, welche der Geschwindigkeit des Fahrzeugs entspricht, gesetzt. Dazu wird die Distanz des zu kollidierenden Objekts und dem Fahrzeug ermittelt. Mit der Distanz wird das Fahrzeug dann entsprechend abgebremst und wenn nötig auch zum Stillstand gebracht. Um nicht in das vorherfahrende Objekt zu fahren, wird die Hälfte der Länge des Fahrzeugs noch von der Distanz abgezogen. Damit sich das Fahrzeug nicht selbst als kollidierendes Objekt erkennt, muss der Layer des Fahrzeugs für kurze Zeit geändert werden.

\begin{lstlisting}[caption={Erkennen von anderen Fahrzeugen und Hindernissen},label={lst:Hinderniss_erkennen}]
private void checkRaycast()
{
	// Save current object layer
	int oldLayer = gameObject.layer;
	//Change object layer to a layer it will be alone
	gameObject.layer = 12;
	int layerToIgnore = 1 << 12;
	layerToIgnore = ~layerToIgnore;

	RaycastHit[] hits;		
	hits = Physics.RaycastAll(transform.position, transform.forward, raycastSize, layerToIgnore);
	bool somethingInFront = false;
	for(int i = 0; i < hits.Length; i++)
	{
		RaycastHit hit = hits[i];
		GameObject collidedObject = hit.collider.gameObject;
		if(collidedObject.name.Equals("jeep(Clone)") || collidedObject.name.Equals("Rock(Clone)"))
		{
			float distance = getDistance(gameObject, collidedObject);
			if(gameObject.name.Equals("jeep(Clone)"))
			{
				distance = distance - (lengthCar / 2 + 1f);
			}
			else
			{
				distance = distance - (lengthTanker / 2 + 1f);
			}
			brakeWithDistance(distance);
			mayIdrive = false;
			somethingInFront = true;
		}
		if(somethingInFront == false)
		{
			mayIdrive = true;
		}
	}
	raycastSize = speed;
	if(raycastSize < 1)
	{
		raycastSize = 1;
	}
	// set the game object back to its original layer
	gameObject.layer = oldLayer;
}
\end{lstlisting}

\section{Ampelsteuerung}
\label{Ampelsteuerung}

Die Ampelsteuerung wurde als eigene Applikation entwickelt. Die Ampelsteuerung kann grundsätzlich unabhängig von der Unity-Spielwelt laufen, auch die Ampeln schalten dementsprechend autonom, auch wenn die Simulation schon beendet sein sollte.

%... evtl. mike wos

Die Ampelsteuerung ist auch mit Mono lauffähig. Mono ist eine C\#-Implementierung für unixoide Betriebssysteme. So läuft die Ampelsteuerung auch mit Linux. Auch eine Kompilierung ist möglich, mit \texttt{xbuild solution.sln} wird eine ausführbare Datei erzeugt, die sowohl mit Linux als auch Windows lauffähig ist.

In Abbildung \ref{img:ampel} ist die Ausgabe der Ampelsteuerung zu sehen. Da die Ampelsteuerung gut funktionierte, wurde in die Ausgabe weniger Zeit investiert, sodass hier nur die aufrufende IP ausgegeben wird.

\begin{figure}[H]
\begin{center}
	\includegraphics[width=0.65\textwidth]{BilderAllgemein/ampelserver.png}
\end{center}
	\caption{Ausgabe Ampelserver}
	\label{img:ampel}
\end{figure}

\section{Erstellen und Löschen von Hindernisse}

Um im Environment Hindernisse zur Laufzeit zu Erstellen, muss dem Terrain-GameObject ein Skript (Obstacles.cs) hinzugefügt werden. In diesem Skript findet die Abarbeitung des Inputs des Users statt.

Wie bereits in \ref{Hindernisse} erläutert wird zum Start der Verkehrssimulation das Hindernis geladen. Dies erfolgt über den "Load" Befehl.

\begin{lstlisting}[caption={Laden des Hindernisses},label={lst:Hinderniss_laden}]
private GameObject prefabLog;
void Start()
{
	prefabLog = Resources.Load("Rock", typeof(GameObject)) as GameObject;
}
\end{lstlisting}

Der in \ref{Hindernisse} beschrieben soll ein Button gedrückt gehalten werden um Hindernisse spawnen zu können. Dieser wird über einen KeyCode definiert. Im Skript wird nun in jedem Update Aufruf darauf gewartet, ob der definierte Button gedrückt und ein "MouseDown" Event vorkommt. Mit diesem Event kann die Position der Maus auf dem Bildschirm herausgefunden werden, jedoch stimmen diese nicht mit den Welt-Koordinaten überein. Deshalb muss hier eine Umwandlung durchgeführt werden, welche mit Hilfe eines Raycasts gelöst wurde. Durch das Instanzieren wird das neu erstellte Hindernis in der Welt platziert.

\begin{lstlisting}[caption={Erstellen des Hindernisses},label={lst:Hinderniss_erstellen}]
private KeyCode shiftLeft = KeyCode.LeftShift;
if(Input.GetMouseButtonDown(0) && Input.GetKey(shiftLeft)) //Left mouse button clicked
{
	Vector3 mousePosition = Input.mousePosition;
	var ray = Camera.main.ScreenPointToRay(mousePosition);
	RaycastHit hit;
	if(Physics.Raycast(ray, out hit, 1000f))
	{
		Vector3 position = hit.point;			Vector3 yOffset = new Vector3(0, 1.5f, 0);
		position += yOffset;			
		GameObject prefabInstance = Instantiate(prefabLog, position, new Quaternion()) as GameObject;
	}
}
\end{lstlisting}

Zum Löschen eines Hindernisses muss die rechte Maustaste in Kombination mit dem vorher definierten Button verwendet werden. Mittels "Destroy" wird anschließend das erkannte GameObject wieder von der Welt gelöscht.

\begin{lstlisting}[caption={Zerstören des Hindernisses},label={lst:Hinderniss_zerstören}]
if(Input.GetMouseButtonDown(1) && Input.GetKey(shiftLeft)) //Right mouse button clicked
{
	Vector3 mousePosition = Input.mousePosition;
	var ray = Camera.main.ScreenPointToRay(mousePosition);
	RaycastHit hit;
	if(Physics.Raycast(ray, out hit, 1000f))
	{
		Vector3 position = hit.point;
		GameObject collidedObject = hit.collider.gameObject;
		if(collidedObject.name.Equals("Rock(Clone)"))
		{
			Destroy(collidedObject);
		}
	}
}
\end{lstlisting}

\begin{figure}[H]
\begin{center}
\includegraphics[width=0.5\textwidth]{BilderAllgemein/rock.PNG}
\end{center}
	\caption{Hindernis}
	\label{img:hindernis}
\end{figure}

\section{Aus- und Einfahren von Fahrzeugen anderer Gruppen}
\label{Aus-_und_Einfahren_von_Fahrzeugen_anderer_Gruppen}

Mit dem Plugin `Unity3D.Amqp` (\url{https://github.com/CymaticLabs/Unity3D.Amqp}) für Unity ist es möglich einen RabbitMQ-Server direkt in Unity einzubinden.

Die Konfiguration der Serverdaten werden dabei direkt in den Menüeinstellungen von Unity vorgenommen. Die verwendete Konfiguration ist in Abbildung \ref{img:rabbit} zu sehen.

\begin{figure}[H]
\begin{center}
	\includegraphics[width=0.9\textwidth]{BilderAllgemein/rabbitconfig.png}
\end{center}

	\caption{Einstellungen RabbitMQ}

	\label{img:rabbit}
\end{figure}

Jede Gruppe bekam einen eigenen Benutzer samt Passwort. Damit man Nachrichten empfangen kann, muss man sich auf eine `Queue` subscriben.

Das Plugin stellt anschließend Methoden zur Verfügung. Eine solche Methode ist \texttt{OnMessageReceived()}. Damit lässt sich eine eingehende Nachricht an ein Objekt in der Spielwelt weitergeben. Das Objekt kann daraufhin reagieren, beispielsweise ein Auto erzeugen.

\begin{figure}[H]
\begin{center}
	\includegraphics[width=0.9\textwidth]{BilderAllgemein/rabbitqueue.png}
\end{center}

	\caption{RabbitMQ Queue}

	\label{img:rabbitq}
\end{figure}

In Abbildung \ref{img:rabbitq} sieht man wie eine eingehende Nachricht beim Objekt \texttt{SpawnField} eine Methode aufruft.
% !TeX root = Trafficsimulation_Documentation.tex

\chapter{Review einer fremden Architekturdokumentation}
\label{Review_einer_fremden_Architekturdokumentation}

\thispagestyle{standard}
\pagestyle{standard}

In diesem Kapitel wird die Architekturdokumentation der Gruppe Binna/Dorfer/Gruber/Mühlbacher/Wieser überprüft.
Dazu wurde die uns zur Verfügung gestellte Version 8 vom 27.06.2017 reviewt. Als Hilfestellung wurden auch Vorgaben aus den im Labor erwähnten Podcasts und Dokumente zum Thema Architektur und Architektur-Reviews genutzt.

Als Grundlage für die Architekturdokumentation wurde das arc42 Template verwendet. Dadurch ließ sich beim Review leicht feststellen, wie weit die einzelnen Punkte der Vorlage erfüllt wurden, oder wo es Abweichungen gab.


Der Abschnitt Einführung und Ziele enthält die Aufgabenstellung und alle geforderten Ziele. Dabei werden diese einfach und verständlich dokumentiert. Detailliertere Beschreibungen folgen erst in den nächsten Abschnitten.


Die tabellarische Form der Anforderungen/Qualitätsziele/Stakeholder ergibt eine übersichtliche Zusammenfassung. Allerdings sind manche Punkte verteilt an mehreren Stellen im Dokument beschrieben und es könnten daher Dinge übersehen werden. z.B. "Der User soll gewisse Parameter die Simulation betreffend verändern können.". Welche Parameter genau veränderbar sein sollen, findet man aber woanders.


Auch bei den Rahmenbedingungen wurde eine Tabelle genutzt die sicherstellt, dass diese eindeutig vom Leser erkennbar sind.


Im fachlichen Teil des Abschnitts Kontextabgrenzung fehlt die Beschreibung des Use-Case Diagramms. Nicht mit dem Projekt vertraute Personen könnten dieses falsch interpretieren. Der technische Kontext ist aufgrund der vorhergehenden Beschreibung besser zu verstehen. Eventuell wäre es noch sinnvoll zu beschreiben, was das System NICHT erfüllen soll.


Die Lösungsstrategien sind teilweise identisch mit den Rahmenbedingungen und geben keine Wirkliche Strategie vor.
Außerdem verhindert die Aufzählungsform eine Begründung der einzelnen Punkte. Leser werden nicht darüber informiert wieso es zu den einzelnen Entscheidungen kam und könnten dieselben Fragen erneut Stellen.


Die Bausteinsicht ist übersichtlich gegliedert, indem sie in Ebenen mit verschiedenen Detailgraden unterteilt wird. Allerdings fehlt die Beschreibung zum Komponentendiagramm in der höchsten Ebene die als Übersicht über die einzelnen Komponenten für den Leser dienen soll. Gerade an dieser Stelle wären allgemeine Informationen und Beschreibungen wichtig. In der zweiten Ebene wurden die einzelnen Komponenten ausführlicher Beschrieben, allerdings fehlen einige technische Details in Textform (z.B. wie einzelne Schnittstellen technisch realisiert werden).


Grundsätzlich wurde in der Architekturdokumentation zu sehr davon ausgegangen, dass der Leser bereits über das Projekt informiert ist. Auf Hilfen wie Anforderungsschablonen oder Patterns wurde nicht zurückgegriffen. Laufzeit- und Verteilungssicht fehlen. Kurz vor dem Abgabetermin stand nur die Version 8 der Architekturdokumentation zur Verfügung. Das Review basiert daher nur auf den bis zu diesem Zeitpunkt dokumentierten Punkten des arc42 Templates.
%{\renewcommand{\setseparator}{ \and }
%	\title{Versionshistorie}
%	\author{\vhListAllAuthorsLongWithAbbrev}
%	\date{Version \vhCurrentVersion\ from \vhCurrentDate}
%	\maketitle
%}

\newcommand{\docTitle}{An example for vhistory}
\hypersetup{%
	pdftitle  = {\docTitle},
	pdfkeywords = {\docTitle, Version \vhCurrentVersion
		from \vhCurrentDate},
	pdfauthor = {\vhAllAuthorsSet}
}

\begin{versionhistory}
  
\vhEntry{0}{03.04.2017}{FS}{Dokumentation erstellt (Vorlage: Martin Uray)}
\vhEntry{0.1}{15.04.2017}{MT}{Komponentenbeschreibung}
\vhEntry{0.2}{08.06.2017}{AR}{Erweiterung auf Arc42 Template}
\vhEntry{0.3}{10.06.2017}{LA}{Hinzufügen \ref{Laufzeitsicht} und \ref{Verteilungssicht}}
\vhEntry{0.4}{15.06.2017}{FS}{Erweiterung Ampelserver}
\vhEntry{0.5}{18.06.2017}{LA}{Hinzufügen der Messaging Komponente}
\vhEntry{0.6}{15.06.2017}{AR}{Fahrzeuglogik, Erkennen von Hindernissen}
\vhEntry{0.7}{01.07.2017}{PR}{Review}
\vhEntry{0.8}{12.07.2017}{LA}{Hinzufügen \ref{Fahrzeugsteuerung} und \ref{Erstellung_der_Welt}}

\end{versionhistory}

Autoren:

\vhListAllAuthorsLongWithAbbrev

%\include{DokumentPraktikumsbericht}
%\include{40PraktischeUmsetzungGISControl}
%\include{50RueckundAusblick}


%%%%%%%%%%%%%%%%%%%%%%%%%%%%%%%%%%%%%%%%%%%%%%%%%%%%%%%%%%%%%%%%%%%%%%%%%%%%%%%%%%%%%%%%%%%%%%%%%%%%%%%%%%%%
% LITERATURVERZEICHNIS

\interlinepenalty=10000 % Literatureinträge: Absätze zusammenhalten
\clearpage
\addcontentsline{toc}{chapter}{Literatur}
\singlespace
%\bibliography{12bibliografie}
%\bibliography{Literaturverzeichnis}
\interlinepenalty=100

%%%%%%%%%%%%%%%%%%%%%%%%%%%%%%%%%%%%%%%%%%%%%%%%%%%%%%%%%%%%%%%%%%%%%%%%%%%%%%%%%%%%%%%%%%%%%%%%%%%%%%%%%%%%
% ANHÄNGE

%\begin{appendix}
%\include{Anhang_Mathematik}                 % A
%\include{Anhang_FormatDerParameterdateien}  % B
%\include{Anhang_Quelltexte}                  % C
%\include{Anhang_Datenblaetter}              % D
%\include{Anhang_Glossar}                    % E
%\end{appendix}

\end{document}