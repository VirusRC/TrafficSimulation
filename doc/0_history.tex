%{\renewcommand{\setseparator}{ \and }
%	\title{Versionshistorie}
%	\author{\vhListAllAuthorsLongWithAbbrev}
%	\date{Version \vhCurrentVersion\ from \vhCurrentDate}
%	\maketitle
%}

\newcommand{\docTitle}{An example for vhistory}
\hypersetup{%
	pdftitle  = {\docTitle},
	pdfkeywords = {\docTitle, Version \vhCurrentVersion
		from \vhCurrentDate},
	pdfauthor = {\vhAllAuthorsSet}
}

\begin{versionhistory}
  
\vhEntry{0}{03.04.2017}{FS}{Dokumentation erstellt (Vorlage: Martin Uray)}
\vhEntry{0.1}{15.04.2017}{MT}{Komponentenbeschreibung}
\vhEntry{0.2}{08.06.2017}{AR}{Erweiterung auf Arc42 Template}
\vhEntry{0.3}{08.06.2017}{LA}{Hinzufügen der Messaging Komponente}
\vhEntry{0.4}{15.06.2017}{FS}{Erweiterung Ampelserver}
\vhEntry{0.4}{15.06.2017}{AR}{Fahrzeuglogik, Erkennen von Hindernissen}


\end{versionhistory}

Autoren:

\vhListAllAuthorsLongWithAbbrev
