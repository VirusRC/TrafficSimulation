% !TeX root = Trafficsimulation_Documentation.tex

\chapter{Design}
\label{Design}

In diesem Kapitel werden die Implementierungen der einzelnen Komponenten sowie deren Schnittstellen zu anderen Komponenten dokumentiert.

\thispagestyle{standard}
\pagestyle{standard}

\section{Strassensystem}
\label{Strassensystem}

\subsection{Ampeln}

\subsection{Strassen}

\subsection{Kreuzungen}

\section{Fahrzeuglogik}
\label{Fahrzeuglogik}

\section{Ampelsteuerung}

\section{Hindernisse}
\label{Hindernisse}

Mit einem Klick der Maus sollen überall in der Welt Hindernisse platziert werden, welche anschließend von Fahrzeugen umfahren werden müssen. Durch einen weiteren Klick auf ein Hindernis soll dieses wieder gelöscht werden. 

Unity ermöglicht das Erstellen von GameObjects zur Laufzeit. Durch den Klick auf die Welt können die Koordinaten des Klicks festgestellt werden und somit das GameObject, welches bereits beim Start der Verkehrssimulation geladen wurde, platziert werden.

Durch das Halten eines Buttons auf der Tastatur soll die Erstellung von Hindernissen aktiviert werden. Dies verhindert, dass der Benutzer unbeabsichtigt zu viele Hindernisse erstellt.

\section{Aus- und Einfahren von Fahrzeugen anderer Gruppen}