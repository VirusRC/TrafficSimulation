\section{Komponenten}
Das Projekt setzt sich aus vier Komponenten zusammen. Zentrale Komponente ist die Bankmanagement Komponente, die die Basisfunktionalitäten eines Banksystems zur Verfügung stellt. Sie wird ergänzt um einen CSV Handler, der für die Datenhaltung des Banksystems verwendet wird, sowie um eine Logger Komponente, mit deren Hilfe Logger Messages generiert werden können. Des Weiteren wurde eine CurrencyConverter Komponente für die Umrechnung von Währungskursen erstellt.


\subsection{BankManagment}
Die Bankmanagement Komponente bietet die wesentlichen Funktionalitäten einer Bank, die in der Aufgabenstellung verlangt waren, an. Sie setzt sich aus zehn Klassen zusammen. Für die Abbildung der Kunden-, Konten- und Transaktionsdaten wurden die Klassen Account, AccountType, CurrencyClass, Customer, Disposer, Transaction und TransactionType implementiert. In einer weiteren Klasse, der BankmanagementClass, werden die Objekte dieser Klassen in Vektoren verwaltet, wobei die Vektoren jeweils über CSV Files mit Daten befüllt werden können. Die Klasse Validator beinhaltet Hilfsfunktionen, die dazu dienen, Eingaben wie z.B. die Übergabe der Wohnadresse eines Kunden auf deren Richtigkeit zu überprüfen. Die Schnittstellenfunktionen, die die Bankmanagement Komponente zur Verfügung stellt, werden im Folgenden beschrieben.

Um die Funktionalität des Bankensystems in seine Applikation aufnehmen zu können, muss die in der Solution BankManagment generierte DLL in das eigene eingebunden werden.

%-------------------------------------------------------------
\subsection{CSV Handler}
Der CSV Handler bietet Funktionen zur Persisitierung des Datenbestandes an. Die Daten werden in Form von csv-Dateien gespeichert. Insgesamt gibt es 5 CSV-Dateien zum speichern der Daten:
\begin{itemize}
	\item customer.csv\\ (ID;Vorname;Nachname;Strasse;Hausnummer;Postleitzahl;Ort;Staat;Customer Active Flag)
	\item account.csv\\ (ID;IBAN;Währung;Erstelldatum;Kontostand;Zinsen;Überziehungsrahmen;Kontotyp;Account Active Flag)
	\item transactions.csv\\ (ID;Betrag;von ID Konto;zu ID Konto;Transaktionstyp)
	\item accountTypes.csv\\ (ID;Kontotyp;Kontotyp Kürzel)
	\item transactionTypes.csv (ID;Transaktionstyp;Transaktionstyp Kürzel)
	\item disposers.csv\\ (ID;Verfüger Konto ID;Verfüger Kunden ID)
\end{itemize}
Darin werden die Daten als Strings gespeichert. Jede Zeile repräsentiert z.B. einen Customer. Die Trennung der einzelnen Datenfelder eines Eintrags erfolgt mittels Semikolon.Um die csv Dateien lesen und schreiben zu können wurde die Komponente CSV Handler implementiert. Gelesen wird immer das gesamte csv File. Die Daten werden von den Schnittstellen zum Lesen der Datei für die weitere Verarbeitung in Form eines Arrays von Strukturen zur Verfügung gestellt.\\
 Die Persisitierung erfolgt immer in Form des gesamten Datensatzes. Dazu wurden Schnittstellenfunktionen zum schreiben der jeweiligen csv-Datei implementiert. Als Parameter nehmen diese ein Array von Strukturen des jweiligen Typs (CUSTOMER,ACCOUNT,ACCOUNTTYPE,TRANSACTIONTYPE,DISPOSER) sowie die Anzahl der darin enthaltenen Datensätze entgegen. Es muss immer der gesamte Datenbestand zur Persistierung übergeben werden.\\
 Neben den Schnittstellen zum lesen und schreiben der Dateien, gibt es auch Schnittstellen zum Auslesen der letzten vergebenen ID innerhalb einer der csv-Dateien.
%-------------------------------------------------------------
\subsection{CurrencyConverter}
Der CurrencyConverter bietet die Möglichkeit Geldbeträge dreier Währungen umzurechnen. Unterstützt werden 
\begin{itemize}
	\item Euro
	\item US-Dollar
	\item Britische Pfund
\end{itemize}

Um alle Konversionen richtig zu unterstützen, sind für jede Konversionsrichtung eigene Kurse definiert, welche über die zugehörigen Schnittstellenfunktionen geändert werden können.

%-------------------------------------------------------------
\subsection{Logger}
Der Logger ermöglicht es Nachrichten mit verschiedenen Leveln entweder auf \textit{stdout} oder in ein File ausgeben zu lassen. Die Ausgabe auf die jeweiligen Ziele muss jeweils explizit gefordert werden. Der Pfad der Log-Datei kann mit einer eigenen Funktion geändert werden. Standardmäßig liegt diese im Projektverzeichnis an der Wurzel.

Um die Log-Nachrichten einer Kategorie zuweisen zu können sind folgende Log-Level eingeführt:

\begin{itemize}
	\item Trace
	\item Debug
	\item Info
	\item Warn
	\item Error
	\item Fatal
\end{itemize}

Um nur gewisse Log-Nachrichten in der Ausgabe anzuzeigen, kann mithilfe einer bereitgestellten Funktion ein Level definiert werden ab welchem die Nachrichten angezeigt werden.