% !TeX root = Trafficsimulation_Documentation.tex

\chapter{Review einer fremden Architekturdokumentation}
\label{Review_einer_fremden_Architekturdokumentation}

\thispagestyle{standard}
\pagestyle{standard}

In diesem Kapitel wird die Architekturdokumentation der Gruppe Binna/Dorfer/Gruber/Mühlbacher/Wieser überprüft.
Dazu wurde die uns zur Verfügung gestellte Version 8 vom 27.06.2017 reviewt. Als Hilfestellung wurden auch Vorgaben aus den im Labor erwähnten Podcasts und Dokumente zum Thema Architektur und Architektur-Reviews genutzt.

Als Grundlage für die Architekturdokumentation wurde das arc42 Template verwendet. Dadurch ließ sich beim Review leicht feststellen, wie weit die einzelnen Punkte der Vorlage erfüllt wurden, oder wo es Abweichungen gab.


Der Abschnitt Einführung und Ziele enthält die Aufgabenstellung und alle geforderten Ziele. Dabei werden diese einfach und verständlich dokumentiert. Detailliertere Beschreibungen folgen erst in den nächsten Abschnitten.


Die tabellarische Form der Anforderungen/Qualitätsziele/Stakeholder ergibt eine übersichtliche Zusammenfassung. Allerdings sind manche Punkte verteilt an mehreren Stellen im Dokument beschrieben und es könnten daher Dinge übersehen werden. z.B. "Der User soll gewisse Parameter die Simulation betreffend verändern können.". Welche Parameter genau veränderbar sein sollen, findet man aber woanders.


Auch bei den Rahmenbedingungen wurde eine Tabelle genutzt die sicherstellt, dass diese eindeutig vom Leser erkennbar sind.


Im fachlichen Teil des Abschnitts Kontextabgrenzung fehlt die Beschreibung des Use-Case Diagramms. Nicht mit dem Projekt vertraute Personen könnten dieses falsch interpretieren. Der technische Kontext ist aufgrund der vorhergehenden Beschreibung besser zu verstehen. Eventuell wäre es noch sinnvoll zu beschreiben, was das System NICHT erfüllen soll.


Die Lösungsstrategien sind teilweise identisch mit den Rahmenbedingungen und geben keine Wirkliche Strategie vor.
Außerdem verhindert die Aufzählungsform eine Begründung der einzelnen Punkte. Leser werden nicht darüber informiert wieso es zu den einzelnen Entscheidungen kam und könnten dieselben Fragen erneut Stellen.


Die Bausteinsicht ist übersichtlich gegliedert, indem sie in Ebenen mit verschiedenen Detailgraden unterteilt wird. Allerdings fehlt die Beschreibung zum Komponentendiagramm in der höchsten Ebene die als Übersicht über die einzelnen Komponenten für den Leser dienen soll. Gerade an dieser Stelle wären allgemeine Informationen und Beschreibungen wichtig. In der zweiten Ebene wurden die einzelnen Komponenten ausführlicher Beschrieben, allerdings fehlen einige technische Details in Textform (z.B. wie einzelne Schnittstellen technisch realisiert werden).


Grundsätzlich wurde in der Architekturdokumentation zu sehr davon ausgegangen, dass der Leser bereits über das Projekt informiert ist. Auf Hilfen wie Anforderungsschablonen oder Patterns wurde nicht zurückgegriffen. Laufzeit- und Verteilungssicht fehlen. Kurz vor dem Abgabetermin stand nur die Version 8 der Architekturdokumentation zur Verfügung. Das Review basiert daher nur auf den bis zu diesem Zeitpunkt dokumentierten Punkten des arc42 Templates.