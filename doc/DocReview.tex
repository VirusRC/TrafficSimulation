% !TeX root = Trafficsimulation_Documentation.tex

\chapter{Review einer fremden Architekturdokumentation}
\label{Review_einer_fremden_Architekturdokumentation}

In diesem Kapitel wird die Architekturdokumentation der Gruppe Binna/Dorfer/Gruber/Mühlbacher/Wieser überprüft.
Dazu wurde die uns zur Verfügung gestellte Version 8 vom 27.06.2017 reviewt.

\thispagestyle{standard}
\pagestyle{standard}

\section{Review}
\label{Review}

Als Grundlage für die Architekturdokumentation wurde das arc42 Template verwendet. Dadurch ließ sich beim Review leicht feststellen, wie weit die einzelnen Punkte der Vorlage erfüllt wurden, oder wo es Abweichungen gab.


Der Abschnitt Einführung und Ziele enthält die Aufgabenstellung und alle geforderten Ziele. Dabei werden diese einfach und verständlich dokumentiert. Detailliertere Beschreibungen folgen erst in den nächsten Abschnitten. 


Die tabellarische Form der Anforderungen/Qualitätsziele/Stakeholder ergibt eine übersichtliche Zusammenfassung. Allerdings sind manche Punkte verteilt an mehreren Stellen im Dokument beschrieben und es könnten daher Dinge übersehen werden. z.B. "Der User soll gewisse Parameter die Simulation betreffend verändern können." . Welche Parameter genau veränderbar sein sollen, findet man aber woanders.
